
\chapter{Rozdział o czymś tam}

\section{Sekcja A}

W tabeli~\ref{tab:przyk} widzimy przykład tabeli z~nagłówkiem i~odnośnikiem. Tabele tworzymy z~nagłówkiem na górze oraz opcją \texttt{[t]}.
Natomiast na rysunku~\ref{rys:przyk} --- widzimy przykład rysunku z nagłówkiem i~odnośnikiem.
Rysunki tworzymy z nagłówkiem pod spodem oraz opcją \texttt{[b]}.
Rysunki powinny być w formacie PDF; jeśli to niemożliwe, to PNG (w wysokiej rozdzielczości); a~ostatecznie JPG (jak tu). Jeśli chcemy sterować rozmiarem, to zwykle najwygodniej użyć \texttt{width=...}
Ponadto możemy odwoływać się do bibliografii~\cite{a:doc, b:doc}.

Jeśli chodzi o wzory, możemy złożyć je na kilka sposobów, w zależności od potrzeb --- w tekście: $e=\lim_{n\to\infty}\left(1+\frac{1}{n}\right)^n$, wyniesiony do osbnej linii
(warto zwrócić uwagę, że ten i~kolejny są złożone nieco inaczej niż pierwszy):
\[e=\lim_{n\to\infty}\left(1+\frac{1}{n}\right)^n,\] a także wyniesiony z numerem:
\begin{equation}
    e=\lim_{n\to\infty}\left(1+\frac{1}{n}\right)^n.
    \label{wzor:e}
\end{equation}
Do tego oostatniego możemy się odwołać:~\eqref{wzor:e}.

No i oczywiście listingi --- listing~\ref{lst:przyk} pokazuje, jak zrobić to w~miarę poprawnie\ldots{}

\begin{table}[t]
    \begin{center}
        \caption{Przykładowa tabela}\label{tab:przyk}
        \begin{tabular}{l|c|r}
            slkdjfslj  & sdkskd               & s;lkdsdk          \\
            \hline
            slkjd      & skljdsldj            & skljdsjdsldj      \\
            sljkdslkjd & woieupowiepoweiwiewp & weoiw eppowie wpo \\
        \end{tabular}
    \end{center}
\end{table}

\begin{figure}[b]
    \begin{center}
        \includegraphics[width=5cm]{fig/LogoUMCS}
    \end{center}
    \caption{Przykładowy rysunek}\label{rys:przyk}
\end{figure}

\begin{lstfloat}[b]
    \lstset{language=C++}
    \begin{lstlisting}[frame=single]
tab[0:n] = dem[nRows][nCols]; //?
#pragma acc data copy(tab [0:n], slope [0:n])
\end{lstlisting}
    \caption{Jakieś dwie linijki w~C++ (z~OpenACC)}\label{lst:przyk}
\end{lstfloat}

\subsection{Odnosnik do arytykułu}
Artykuł~\cite{bib:art1}
