\documentclass[a4paper,12pt]{book} % nie: report!
\usepackage{url}
% \usepackage{hyperref}

\usepackage{amsfonts} % pakiety od AMS, ułatwiają składanie pewnych techniczno-matematcyznych rzeczy
\usepackage{amsmath} % to do dodatkowych symboli, przydatne
\usepackage{amssymb} % to też do dodatkowych symboli, też przydatne
\usepackage{amsthm}
\usepackage{cmap}
\usepackage{babel}
\selectlanguage{polish}
\usepackage[T1,plmath]{polski} % lepiej to zamiast babel!
\usepackage[utf8]{inputenc} % w razie kłopotów spróbować: \usepackage[utf8x]{inputenc}
\usepackage[
backend=biber,
style=numeric,
sorting=ynt,
defernumbers=true
]{biblatex}

\addbibresource{refs.bib} % plik z bibliografią
\usepackage{fancyhdr} % nagłówki i stopki
\usepackage{indentfirst} % WAŻNE, MA BYĆ!
\usepackage[pdftex]{graphicx} % to do wstawiania rysunków

\usepackage[pdftex,
            left=1.1in,right=1.1in,
            top=1.1in,bottom=1.1in]{geometry} % marginesy
\usepackage{float}
\usepackage[font=small,labelfont=bf]{caption}

\usepackage[colorlinks=true]{hyperref} % odnośniki interaktywne w PDFie
\hypersetup{allcolors=black}

\usepackage{listings}
\lstset{
    basicstyle=\footnotesize\tt,
    numbers=left,
    numberstyle=\tiny,
    frame=tb,
    tabsize=4,
    columns=fixed,
    showstringspaces=false,
    showtabs=false,
    keepspaces,
    commentstyle=\color{red},
    keywordstyle=\color{blue}
}
\newfloat{lstfloat}{htbp}{lolst}[chapter]
\floatname{lstfloat}{Listing}
\def\lstfloatautorefname{Listing}

% jesli potrzeb, można oczywiście wstawić inne pakiety i swoje definicje...



% definicje nagłówków i stopek
\pagestyle{fancy}
\renewcommand{\chaptermark}[1]{\markboth{#1}{}}
\renewcommand{\sectionmark}[1]{\markright{\thesection\ #1}}
\fancyhf{}
\fancyhead[LE,RO]{\footnotesize\bfseries\thepage}
\fancyhead[LO]{\footnotesize\rightmark}
\fancyhead[RE]{\footnotesize\leftmark}
\renewcommand{\headrulewidth}{0.5pt}
\renewcommand{\footrulewidth}{0pt}
\addtolength{\headheight}{1.5pt}
\fancypagestyle{plain}{\fancyhead{}\cfoot{\footnotesize\bfseries\thepage}\renewcommand{\headrulewidth}{0pt}}


% interlinia
\linespread{1.25}



\begin{document}


\begin{titlepage}
    ~

    \begin{tabular}{c@{\hspace{21mm}}|@{\hspace{5mm}}l}
        \vspace{-20mm} &                                                                            \\
        \multicolumn{2}{l}{\hspace{-12.5mm} \includegraphics[width=8cm]{fig/LogoUMCS.jpg}}          \\
        \multicolumn{2}{@{\hspace{20mm}}l}{\vspace{-4mm}}                                           \\
        \multicolumn{2}{@{\hspace{28mm}}l}{\Large \sf UNIWERSYTET MARII
        CURIE-SKŁODOWSKIEJ}                                                                         \\
        \multicolumn{2}{@{\hspace{28mm}}l}{\vspace{-4mm}}                                           \\
        \multicolumn{2}{@{\hspace{28mm}}l}{\Large \sf W LUBLINIE}                                   \\
        \multicolumn{2}{@{\hspace{28mm}}l}{\vspace{-4mm}}                                           \\
        \multicolumn{2}{@{\hspace{28mm}}l}{\Large \sf Wydział Matematyki, Fizyki i
        Informatyki}                                                                                \\
        \multicolumn{2}{@{\hspace{28mm}}l}{\vspace{21mm}}                                           \\
                       & {\sf Kierunek: \textbf{informatyka/matematyka/geoinformatyka/...} }        \\
        %& {\sf Specjalność: \textbf{informatyczna}} \\ % wpisujemy tylko jeśli jest!!!
                       &                                                                            \\\\\\
                       & {\sf \large \bfseries Imię Nazwisko}                                       \\
                       & {\sf nr albumu: \dots}                                                     \\
                       &                                                                            \\\\\\
                       & \Large \sf \bfseries Tytuł po polsku, który zwykle                         \\
                       & \Large \sf \bfseries jest długi na wiele linijek                           \\\\[-10pt]
                       & {\large \sf Title in English}                                              \\
                       & {\large \sf (also a long one)}                                             \\
                       &                                                                            \\
                       &                                                                            \\
                       &                                                                            \\
                       & {\sf Praca magisterska}                                                    \\
                       & \vspace{-7mm}                                                              \\
                       & {\sf napisana w Katedrze ...}                                              \\
                       & {\sf Instytutu ... UMCS}                                                   \\
                       & \vspace{-7mm}                                                              \\
                       & {\sf pod kierunkiem \bfseries stopień/tytuł imię i nazwisko (odmienione!)} \\
        \multicolumn{2}{@{\hspace{28mm}}l}{\vspace{15mm}}                                           \\
        \multicolumn{2}{@{\hspace{28mm}}l}{\textbf{\textsf{Lublin 2022}}}
    \end{tabular}
\end{titlepage}





\sloppy



\thispagestyle{empty}


\newpage{}

\thispagestyle{empty}

\newpage{}



\tableofcontents{}


\chapter*{Wstęp} % z gwiazdką, więc bez numerka...
\addcontentsline{toc}{chapter}{Wstęp} % ...ale w spisie treści ma być

Tu treść wstępu



\chapter{Rozdział o czymś tam}

\section{Sekcja A}

W tabeli~\ref{tab:przyk} widzimy przykład tabeli z~nagłówkiem i~odnośnikiem. Tabele tworzymy z~nagłówkiem na górze oraz opcją \texttt{[t]}.
Natomiast na rysunku~\ref{rys:przyk} --- widzimy przykład rysunku z nagłówkiem i~odnośnikiem.
Rysunki tworzymy z nagłówkiem pod spodem oraz opcją \texttt{[b]}.
Rysunki powinny być w formacie PDF; jeśli to niemożliwe, to PNG (w wysokiej rozdzielczości); a~ostatecznie JPG (jak tu). Jeśli chcemy sterować rozmiarem, to zwykle najwygodniej użyć \texttt{width=...}
Ponadto możemy odwoływać się do bibliografii~\cite{a:doc, b:doc}.

Jeśli chodzi o wzory, możemy złożyć je na kilka sposobów, w zależności od potrzeb --- w tekście: $e=\lim_{n\to\infty}\left(1+\frac{1}{n}\right)^n$, wyniesiony do osbnej linii
(warto zwrócić uwagę, że ten i~kolejny są złożone nieco inaczej niż pierwszy):
\[e=\lim_{n\to\infty}\left(1+\frac{1}{n}\right)^n,\] a także wyniesiony z numerem:
\begin{equation}
    e=\lim_{n\to\infty}\left(1+\frac{1}{n}\right)^n.
    \label{wzor:e}
\end{equation}
Do tego oostatniego możemy się odwołać:~\eqref{wzor:e}.

No i oczywiście listingi --- listing~\ref{lst:przyk} pokazuje, jak zrobić to w~miarę poprawnie\ldots{}

\begin{table}[t]
    \begin{center}
        \caption{Przykładowa tabela}\label{tab:przyk}
        \begin{tabular}{l|c|r}
            slkdjfslj  & sdkskd               & s;lkdsdk          \\
            \hline
            slkjd      & skljdsldj            & skljdsjdsldj      \\
            sljkdslkjd & woieupowiepoweiwiewp & weoiw eppowie wpo \\
        \end{tabular}
    \end{center}
\end{table}

\begin{figure}[b]
    \begin{center}
        \includegraphics[width=5cm]{fig/LogoUMCS}
    \end{center}
    \caption{Przykładowy rysunek}\label{rys:przyk}
\end{figure}

\begin{lstfloat}[b]
    \lstset{language=C++}
    \begin{lstlisting}[frame=single]
tab[0:n] = dem[nRows][nCols]; //?
#pragma acc data copy(tab [0:n], slope [0:n])
\end{lstlisting}
    \caption{Jakieś dwie linijki w~C++ (z~OpenACC)}\label{lst:przyk}
\end{lstfloat}

\subsection{Odnosnik do arytykułu}
Artykuł~\cite{bib:art1}



\chapter{Istniejące rozwiązania}
---
\section{Rozwiązanie 1}
---
\subsection{Rozwiązanie 1.1}
---
\section{Rozwiązanie 2}


\include{src/zalozenia-implementacja.tex}

\chapter{Wdrożenie i testy}


\include{src/wnioski.tex}

\listof{lstfloat}{Spis listingów} % jeśli są listingi
\addcontentsline{toc}{chapter}{Spis listingów}

\listoftables{} % jeśli są tabele
\addcontentsline{toc}{chapter}{Spis tabel}

\listoffigures{} % jeśli są rysunki
\addcontentsline{toc}{chapter}{Spis rysunków}

\printbibheading{}
\printbibliography[heading=subbibliography,nottype=online,title={Artykuły}]{} % wydruk bibliografii
\printbibliography[heading=subbibliography,type=online,title={Odnosniki w sieci}]{} % wydruk bibliografii
\addcontentsline{toc}{chapter}{Bibliografia} % też ręczne dodanie do spisu treści, jak Wstęp
% \bibliographystyle{plain}
% \begin{thebibliography}{99}
%     \bibitem{bib:a} aaaaaaaa
%     \bibitem{bib:b} bbbbbbbb
%     \bibitem{bib:esp32-all-socs} \href{https://www.espressif.com/en/products/socs}(dostęp: 15.04.2023)
% \end{thebibliography}


\end{document}
